%Comment this line out to use real graphics instead of demo boxes.  Do not use \usepackage[demo]{graphicx} because that causes
%conflicts with other packages with dependencies.
\documentclass[sensors,supfile,submit,moreauthors,pdftex]{Definitions/mdpi} 
\usepackage{amsmath}
\usepackage[version=3]{mhchem}
\usepackage{url}
\usepackage{graphicx}
\usepackage{threeparttable}
\usepackage{pdfpages}
\usepackage{multirow}
\usepackage{siunitx}
\usepackage{microtype}
\newcommand{\degree}{$^{\circ}$}

\graphicspath{ {./images/} } %this removes clutter from the root folder

\Title{Background and Properties of Selected Commercially Available, Low-cost Carbon Dioxide and Methane Gas Concentration Sensors}

% Author Orchid ID: enter ID or remove command
\newcommand{\orcidauthorA}{0000-0002-3681-4810} % Add \orcidA{} behind the author's name
\newcommand{\orcidauthorB}{0000-0003-4266-5011} % Add \orcidB{} behind the author's name
%\newcommand{\orcidauthorC}{0000-0003-4266-5011} % Add \orcidC{} behind the author's name

% Authors, for the paper (add full first names)
\Author{Wesley T. Honeycutt $^{1,\dagger,}$\orcidA{}, M. Tyler Ley $^{2,}$, and Nicholas F. Materer $^{3,}$\orcidB{}*}

% Authors, for metadata in PDF
\AuthorNames{Wesley T. Honeycutt, M. Tyler Ley, and Nicholas F. Materer}

% Affiliations / Addresses (Add [1] after \address if there is only one affiliation.)
\address{%
	$^{1}$ \quad Oklahoma State University, Department of Chemistry, 107 Physical Sciences, Stillwater, OK 74078, USA; wes.honeycutt@okstate.edu \\
	$^{2}$ \quad Oklahoma State University Department of Civil Engineering; 319C Engineering South, Stillwater, OK 74078, USA; tyler.ley@okstate.edu \\
	$^{2}$ \quad Oklahoma State University Department of Chemistry; 107 Physical Sciences, Stillwater, OK 74078, USA; materer@okstate.edu
}%

% Contact information of the corresponding author
\corres{Correspondence: materer@okstate.edu}

% Current address and/or shared authorship
\firstnote{Current address: University of Oklahoma Biological Survey; 111 Chesapeake St.; Norman, OK 73019, USA} 
%\secondnote{These authors contributed equally to this work.}
% The commands \thirdnote{} till \eighthnote{} are available for further notes

\begin{document}
	
	\section*{Selection Rational}
	The selection rational, sensing method, and selected properties obtained from the manufacturer documentation is tabulated below.
	%
	Sensors can be generally categorized by detection method, which as optical absorption, chemiresistive (based on the resistance changes of a material due to chemical reaction with an analyte~\cite{wetchakun_semiconducting_2011}), and electrochemical.
	%
	Since studies have cited concerns with electrochemical gas concentration sensors, such as a short lifetime and lack of robustness~\cite{neri_first_2015}, only optical and chemiresistive sensors were selected.
	%
	The potential application of the selected sensors is also impotent in the selection.  For this study, the application includes, but is not limited to, environmental monitoring of local gas concentration and detection of leaks around industrial locations.
	%
	For such applications, the collection of multiple samples at several locations in a given area (possibility including places without power availability) is critical to obtain reliable results.
	%
	Thus, with only two exceptions, sensors selected were all commercially available in large volumes (at least 1000 units) at low-cost (defined here as less than \$100 per unit in bulk)
	%
    The sensors were further selected based on the reported sensitivity at environmental concentrations of \ce{CO2} (around 400 ppm~\cite{blasing_recent_2016,dlugokencky_trends_2016}) and \ce{CH4} ( under 2 ppm~\cite{turner_large_2016,bamberger_spatial_2014,dlugokencky_trends_2016-1}), and at concentrations of several thousand ppm, which simulate a potential leak.
    %
	In the selection process, the cost, limit of detection, precision, accuracy, reliability, and power consumption are all important parameters, many of which are not reported the manufacture or cannot be directly compared to other similar sensors.   
	%% PARA
	
   \section*{Sensing Technology}
  Lower cost optical sensors typically utilize nondispersive infrared (NDIR) sensing.  
  %
  This method utilizes a broad spectrum light source which is restricted by a narrow band pass filter across the absorbance maximum before reaching the detector.
  %
  Since these sensors utilize the Beer-Lambert law to relate absorption to concentration, the calibration is only dependent on the geometry of the sensor and physical properties of the gas~\cite{bacsik_ftir_2004}.
  %
  In general, NDIR detection is utilized for \ce{CO2} due to its relatively large molar absorption coefficient, allowing for short path lengths to be used in devices.  
  %
  For \ce{CH4}, NDIR detection is limited due to its lower absorption coefficient and overlapping symmetric C-H stretches.
  %
  The overlapping stretches makes \ce{CH4} difficult to distinguish from other common aliphatic gases such as ethane and propane~\cite{coblentz_society_inc._evaluated_????}.
  %
  The selected lower cost chemiresistive sensors typicality detect \ce{CH4}  using a thin oxide film~\cite{neri_first_2015} and work by measuring resistance changes due to differences in the electron transport through the metal oxide film, in the presence of oxygen and target gases~\cite{albert_cross-reactive_2000}.
  %
  The resistance change is typically non-linear with the analyte concentration.
  %
  The chemiresistive sensors are known to respond to a range of hydrocarbon gases~\cite{sekhar_development_2016}, which should be considered when integrating these sensors into a sensor platform.
  %% PARA
  
  	\section*{Selected Sensors}
   Table~\ref{tab:CO2property} lists the selected \ce{CO2} sensors with important properties obtained from the manufacturer.
   %
   Table~\ref{tab:CH4property} lists the \ce{CH4} or hydrocarbon sensors and respective properties.  The K-30, COZIR, Dynament, and Telaire sensors are all NDIR sensors.
   %
   These sensors were chosen as low-cost, lightweight sensors with satisfactory detection parameters of \ce{CO2}.
   %
   Dynament also provides a dual gas NDIR sensor (MSH-DP/HC/CO2/) designed to measure both \ce{CO2} and \ce{CH4} concentrations.
   %
   This ability was attractive given low-cost and portability requirements.
   %
   The \ce{CO2} and \ce{CH4} Gascard sensors sold by GHG Analytical were an order of magnitude more expensive than the other chosen NDIR sensors, which have a cost between that of the lowest cost sensors on our list and that of the bench-top analyzers.
   %
   Their specifications combined with the included pressure and temperatures compensation make them attractive enough to make up for the expense.
   %
   In addition to the Gascard sensor, the Dynament hydrocarbon sensors (MSH-P/HC and MSH-DP/HC/CO2/) were chosen as inexpensive candidates for \ce{CH4} detection.
   %
   Chemoresistive sensors include the MQ-4 from Hanwei Electronics and TGS-2600, TGS-2610, and TGS-2611 manufactured by Figaro Engineering Inc. sensors.
   %
   The TGS sensors are used in commercial \ce{CH4} and air quality detectors.
   %
   There are several different MQ versions optimized for hydrocarbon sensing.
   %
   The MQ-4 sensor was chosen as this variant was specifically tuned for \ce{CH4}.
   %% PARA

	\begin{table*}[!th]
		\caption{Manufacturer listed properties of evaluated \ce{CO2} sensors}
		\label{tab:CO2property}
		\small
		\resizebox{\textwidth}{!}{%
			\begin{tabular}{ r | c c c c c }
				Sensor               & Supplier         & Type & Sampling Method   & Cal. Range   & Op. Range   \\ \hline
				K-30 SE-0018         & CO$_2$Meter      & NDIR & flow or diffusion & 0-5000 ppm   & 0-10000 ppm \\
				COZIR AMB GC-020     & CO$_2$Meter      & NDIR & flow or diffusion & 0-5000 ppm   & 0-10000 ppm \\
				Gascard CO$_2$       & GHG Analytical   & NDIR & flow              & 0-50000 ppm  & 0-50000 ppm \\
				MSH-P/CO2/NC/5/V/P/F & Dynament         & NDIR & diffusion         & 0-2491 ppm   & 0-5000 ppm  \\
				MSH-DP/HC/CO2/NC/P/F & Dynament         & NDIR & diffusion         & 100-2500 ppm & 0-5000 ppm  \\
				Telaire T6615        & General Electric & NDIR & flow or diffusion & 0-2000 ppm   & 0-2000 ppm
		\end{tabular}}\\\\\\
		\resizebox{\textwidth}{!}{%
			\begin{tabular}{ r | c c c c c c }
				Sensor               & Warm Up        & T               & Humidity & Auto-cal & V Input      & Avg. I   \\ \hline
				K-30 SE-0018         & \textless1 min & 0-50\degree C   & 0-95\%   & Yes      & 4.5-14 VDC   & 40 mA    \\
				COZIR AMB GC-020     & \textless3 s   & 0-50\degree C   & 0-95\%   & Yes      & 3.25-5.5 VDC & 1.5 mA   \\
				Gascard CO$_2$       & 30 s           & 0-45\degree C   & 0-95\%   & Yes      & 7-30 VDC     & 250 mA   \\
				MSH-P/CO2/NC/5/V/P/F & 45 s           & -20-50\degree C & 0-95\%   & No       & 3.0-5.0 VDC  & 75-85 mA \\
				MSH-DP/HC/CO2/NC/P/F & 45 s           & -20-50\degree C & 0-95\%   & No       & 3.0-5.0 VDC  & 75-85 mA \\
				Telaire T6615        & 10 min         & 0-50\degree C   & 0-95\%   & Yes      & 0-5 VDC      & 33 mA    
		\end{tabular}}
	\end{table*}
	
	\begin{table*}[!th]
		\caption{Manufacturer listed properties of evaluated \ce{CH4} sensors}
		\label{tab:CH4property}
		\small
		\resizebox{\textwidth}{!}{%
			\begin{tabular}{ r | c c c c c }
				Sensor               & Supplier           & Type           & Sampling Method & Cal. Range     & Op. Range \\ \hline
				MQ-4                 & Futurelec          & chemiresistive & diffusion       &                & 200-10000 ppm \\
				Gascard CH$_4$       & GHG Analytical     & NDIR           & flow            & 0-50000 ppm    & 0-50000 ppm \\
				MSH-P/HC/NC/5/V/P/F  & Dynament           & NDIR           & diffusion       & 0-5000 ppm     & 0-10000 ppm \\
				MSH-DP/HC/CO2/NC/P/F & Dynament           & NDIR           & diffusion       & 5000-11000 ppm & 0-10000 ppm\\
				TGS-2600             & Figaro Engineering & chemiresistive & diffusion       &                & 1-30 ppm\\
				TGS-2610             & Figaro Engineering & chemiresistive & diffusion       &                & 1000-25000 ppm\\
				TGS-2611             & Figaro Engineering & chemiresistive & diffusion       &                & 500-10000 ppm
		\end{tabular}}\\\\\\
		\resizebox{\textwidth}{!}{%
			\begin{tabular}{ r | c c c c c c }
				Sensor               & Warm Up       & T               & Humidity & Auto-cal & V Input         & Avg. I          \\ \hline
				MQ-4                 & \tnote{\dag}  &                 &          & No       & 5 VDC           & \textless150 mA \\
				Gascard CH$_4$       & 30 s          & 0-45\degree C   & 0-95\%   & Yes      & 7-30 VDC        & 250 mA          \\
				MSH-P/HC/NC/5/V/P/F  & 30 s          & -20-50\degree C & 0-95\%   & No       & 3.0-5.0 VDC     & 75-85 mA        \\
				MSH-DP/HC/CO2/NC/P/F & 30 s          & -20-50\degree C & 0-95\%   & No       & 3.0-5.0 VDC     & 75-85 mA        \\
				TGS-2600             &  \tnote{\dag} &                 &          & No       & 5.0$\pm$0.2 VDC & 4.2$\pm$4 mA    \\
				TGS-2610             &  \tnote{\dag} &                 &          & No       & 5.0$\pm$0.2 VDC & 5.6$\pm$5 mA    \\
				TGS-2611             &  \tnote{\dag} &                 &          & No       & 5.0$\pm$0.2 VDC & 5.6$\pm$5 mA
		\end{tabular}}
		\begin{tablenotes}
			\small
			\item[\dag]{Sensors with no listed warm-up time required 7-day burn-in time}
		\end{tablenotes}
	\end{table*}
	
	\section*{References }
	%	\bibliographystyle{elsarticle-num}
	\externalbibliography{yes}
	\bibliography{bib}
	
	
\end{document}

